\documentclass[12pt,letterpaper]{article}
\usepackage{mla}
\usepackage{wrapfig}

\begin{document}
\begin{mla}{Paul}{English}{ENG 2010-002}{Professor Argyle}{\today}{\textbf{FAQ: 2012-2017 OCS Oil \& Gas Leasing Program}}

The Bureau of Ocean Management, BOEM, recently released the final plans for the 2012-2017 Outer Continental Shelf Oil and Gas Leasing Program. This document is created, and reviewed by United States officials and approved by the Bureau of Ocean Energy Management. It documents scheduled off-shore land leasing that allows companies to setup drilling rigs.

\begin{enumerate}
\item \textbf{What do the plans cover?}\newline
The 2012-2017 Oil and Gas Leasing Program is a schedule of lease sales that allow companies to provide off-shore drilling in the Outer Continental Shelf regions of the United States. Sizes, times of availability, and locations are all provided in the program. 

\item \textbf{Why do we need this leasing plan? What purposes does it serve?}\newline
The Outer Continental Shelf (OCS) Lands Act protects the undersea land at all areas around the United States. It requires that for any offshore leasing for the purpose of drilling must be included in an approved five year program. By having this plan available at draft, review, and final stages the public are able to take an interest in proposed drilling locations, and can learn about these areas if necessary.

\item \textbf{What areas and regions are involved?}\newline
The Outer Continental Shelf consists of regions of land outside of the Eastern Coast, Western Coast, Gulf of New Mexico, and Alaska. These undersea areas make up the United States portion of the continental shelf. Regions that are included in the 2012-2017 Program include 5 lease sales in the Gulf of Mexico, 5 in Central Gulf, and 2 in the Easter Gulf. Additionally there are 3 lease sales in the Alaskan shelf, including the Chukchi and Beaufort seas. The Pacific and Atlantic regions are not subject to off-shore leasing.

\item \textbf{What is the process for approving this program?}\newline
The program has gone through 3 stages; a draft, a proposed program, and the final program. At each stage it is reviewed, and released to officials and the public. The final program is approved by the Secretary of the Interior only after it has been submitted to the President and Congress for a minimum of 60 days.

\item \textbf{What additional information is included with the leasing program?}\newline
Released along with the leasing program were a Programmatic Environmental Impact Statement and several supplemental supporting documents. The Programmatic EIS attempts to analyze environmental, social, and economic impacts that relate to the 2012-2017 plan. The supporting documents include economic analysis, study into energy alternatives, environmental forecasting, and other similar and related material.

\end{enumerate}

%%%%%%%%%%%%%%%%%%%%%%%%%%%%%

\begin{workscited}

\bibent
United States. Bureau of Ocean Energy Management. Department of the Interior. \textit{Proposed Final Outer Continental Shelf Oil \& Gas Leasing Program 2012-2017.} June 2012. Web. 19 July 2012. $<$http://www.boem.gov/5-year/2012-2017/$>$.

\bibent
United States. Bureau of Ocean Energy Management. Department of the Interior. \textit{Final Programmatic Environmental Impact Statement} June 2012. Web. 19 July 2012. $<$http://www.boem.gov/5-Year/2012-2017/PEIS.aspx$>$.


% http://www.nottheanswer.org/2012/07/01/offshore-drilling-plan-for-2012-2017-released-by-obama-administration/
% http://public.surfrider.org/files/Surfrider\_2012\_2017\_Oil\_Gas\_PEIS\_Jan2012\_Final.pdf
% http://www.boem.gov/5-year/2012-2017/
% http://hotair.com/archives/2012/06/30/the-obama-administrations-five-year-drilling-plan-hint-its-underwhelming/

\end{workscited}
\end{mla}
\end{document}
