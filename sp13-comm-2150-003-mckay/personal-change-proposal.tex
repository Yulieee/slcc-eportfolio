\documentclass[12pt,letterpaper]{article}
\usepackage{mla}
\usepackage{wrapfig}

% overwrite double spacing
\linespread{1}

\begin{document}
\begin{mla}{Paul}{English}{COMM 1010-003}{Shirene
    McKay}{\today}{\textbf{Personal Change Proposal}}

% do the communication competency self-evaluation
% communication competency score chart

\section{Competency Area}

% define the communication competency you'd like to work on
Interpersonal Interest: This dimension assesses the extent to which
you are likely to initiate and maintain friendship with people who
might be different from you. It also measures how inclined you are to
actively seek out others, as well as your ability to engage them in
interesting conversation (``GCI'').

\section{Goal}

% 1-2 sentences describing goal
% - achievable
% - identify one specific ineffective habit to overcome

I would like to improve my interpersonal interest by seeking out other
acquaintances who I might not normally befriend or connect with.
Specifically I would like to find interests outside of class work with
at least one individual in my classes this semester.

\section{Rationale}

% - few sentences
% - explain why that habit is ineffective
% - describe some specific instances
% - identify at least one negative consequence
% - identify one positive consequence you could gain by improving
I'm not averse to meeting new people, and I often enjoy the company of
others, however I rarely go out of my way to make introductions or
make small-talk with others in the hopes of finding mutual interest.
Either I'm introduced, or I seek out those who I've already found share
a mutual interest, or I'm simply forced to interact with them for a
prolonged period of time. I'm usually humbly content with my casual
social network as it is. I rarely, if ever, seek out interpersonal
relationships because of differences of culture.

The potential rewards in interpersonal, or intercultural,
relationships are tremendous. Including acquiring knowledge about the
world, breaking stereotypes, and acquiring new skills (``Martin'' 391). It
would be a shame to waste this time amongst hundreds of fellow
students, and not take part in as much shared experience as possible.

\section{Implementation}

% - when you will begin putting your changes into practice
% - how you will set the stage to have opportunity to work on chosen
%   strategies
% - how will you motivate yourself to do it?

I should start now, with every class period, with every trip to the
library. I'll introduce myself to at least one new person in each
course I take, and I'll make sure to ask at least one question that
doesn't relate directly to the course at hand, or too something of my
own personal interest.

Though I'm likely to find many shared interests amongst my peers, I'll
seek out someone who can authentically educate me on something I've
never experienced before.

\pagebreak
\section{Works Cited}

\bibent
GCI, Communication 2150 Course Readings. SLCC, Spring 2013.

\bibent
Martin, Judith N., and Thomas K. Nakayama. ``Chapter 10: Culture,
Communication, and Intercultural Relationships.''
\textit{Intercultural Communication in Contexts.} 6th ed. Boston: 
McGraw-Hill, 2013. 391. Print.

\end{mla}
\end{document}

