\documentclass[12pt,letterpaper]{article}
\usepackage{mla}
\usepackage{wrapfig}

% overwrite double spacing
\linespread{1}

\begin{document}
\begin{mla}{Paul}{English}{COMM 1010-003}{Shirene
    McKay}{\today}{\textbf{Personal Change Progress Report}}

\section{Competency Area}

% identify the competency you selected in your change proposal
Interpersonal Interest: This dimension assesses the extent to which
you are likely to initiate and maintain friendship with people who
might be different from you. It also measures how inclined you are to
actively seek out others, as well as your ability to engage them in
interesting conversation (``GCI'').

\section{Goal}

% Include a description of the goal you set for yourself this semester
I would like to improve my interpersonal interest by seeking out other
acquaintances who I might not normally befriend or connect with.
Specifically I would like to find interests outside of class work with
at least one individual in my classes this semester.

\section{Chapter 5: Identity and Intercultural Communication}
\subsection{Tactic}
% list the specific tactic/communication behavior you are applying to
% help you achieve your goal/competency. Include in text citations
% from each chapter
Focus on learning more about the identity of the people I work and
interact with on a regular basis by asking honest questions that may
help me learn about their identity.

\subsection{Implementation}
% - how are you applying the tactic to your own relationships? Draw
% from you journal entries to help illustrate with specific examples
% - how is this change being received by the other person(s) in the
% relationship?
% - what are the positive consequences of this new behavior?
% - what problems, frustrations or set-backs are you experiencing with
% your new bahavior?
By asking what kind of culture do they belong too, or asking about
choices they've made that might seem curious we can learn more about
the one or many identities that build up into the behaviors that make
them who they are. Do they have strong personal identity? Are they
from a culture that gives them a lot of influence? Are they from a
differing region or are they of a different ethnicity?

\subsection{Recommendations}
% outline your plans for future interactions
% - will you continue your present course of actions? why? why not?
% - will you modify your actions or change your tactic? why?
% - support you recommendations with citations from the text.
Keeping a strong interest in the identity of others usually requires
more than just guesses based on externally observable stereotypes, and
although these can help be good indicators for particular identity
connections early on, deep relationships require more information to
get to know what affects a persons identity. In the future continuing
to have an active interest in people, can help improve my
understanding of others identity.

\section{Chapter 6: Language and Intercultural Communication}
\subsection{Tactic}
Pay attention to the language people use as a way of getting to know
them or the culture they belong too. Specifically try to understand or
ask about high-context or low-context communication when you recognize
examples of it.

\subsection{Implementation}
Recognizing high context communication is a bit of a trick when your
goal is to inquire about it. The whole point of the high context
communication is not having to dull yourself with the details or
intracacies, and still be able to make a point. Still clarifying or
asking about high-context details helps to understand how a person
uses language or the world around them to communicate and how they
wish to be perceived.

\subsection{Recommendations}
It's important not to over represent a persons use of language, it's
easy to apply meaning that may or may not actually be there to
conversations you're having. Some people may in fact use a lot of
metaphor, and have subtle overt meaning in their language and
behavior. Others may seem like they are a high-context communicator,
when in fact they may not have additional meaning to attach to the
words they say.

\section{Chapter 7: Nonverbal Codes and Cultural Space}
\subsection{Tactic}
Simply displaying an obvious non-verbal action can illicit a
non-verbal response in many cases, including smiling regularly. By
smiling a few times a day even just in passing, while 
working, or in brief encounters I can gain information about the
non-verbal styles of those around me.

\subsection{Implementation}
Simply working to share my feelings through smiling or telling outside
reactions. Greeting friends and coworkers, or in making new
introductions I make sure to either smile or be receptive to the other
person. Likewise it's important to listen to the non-verbal queues
that others around you might share, and act appropriately.

\subsection{Recommendations}
Non-verbal communication can be wildly important in our interactions
with others, however we can't always just shrug and gesture our way
into a happy or successful life, so it's important to augment
non-verbal communication with clear understood communication, often
verbal, as well. By doing so we help to prevent misrepresentations of
how we feel, or how we receive others.

\pagebreak
\section{Works Cited}

\bibent
Martin, Judith N., and Thomas K. Nakayama. ``Chapter 10: Culture,
Communication, and Intercultural Relationships.''
\textit{Intercultural Communication in Contexts.} 6th ed. Boston: 
McGraw-Hill, 2013. 391. Print.

\bibent
GCI, Communication 2150 Course Readings. SLCC, Spring 2013.

\end{mla}
\end{document}

