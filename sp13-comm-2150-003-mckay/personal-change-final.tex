\documentclass[12pt,letterpaper]{article}
\usepackage{mla}
\usepackage{wrapfig}

% overwrite double spacing
\linespread{1}

\begin{document}
\begin{mla}{Paul}{English}{COMM 1010-003}{Shirene
    McKay}{\today}{\textbf{Personal Report Final}}

\section{Overview}

This article presents several observations on journal articles created throughout the course of this class. Sharing in content the tactics, and 

\section{Competency Area}

\textbf{Interpersonal Interest:} This dimension assesses the extent to which
you are likely to initiate and maintain friendship with people who
might be different from you. It also measures how inclined you are to
actively seek out others, as well as your ability to engage them in
interesting conversation (``GCI'').

\section{Goal}

To improve communication with other individuals by working on social interest and connectedness in conversation and general interest.

\section{Chapter 8: Transitions}

\subsection{Tactic}

To find situations where intercultural transitions occur. If found to probe, and see if I can learn more about the feelings and type of cultural transition that has occurred. People can face culture shock, issues when experiencing assimilating, and can face other issues such as segregation, I think it may be enlightening to ask about these situations where they might have occurred (``Martin, and Nakayama''). 

\subsection{Implementation}

I simply observed situations where cultural transitions in peoples lives were apparent. This included a friends change of career, and the very literal transition station of a relay race. By taking part actively in transitions, and inquiring about observable transitions, I was able to learn quite a bit, and gain a greater appreciation for certain aspects of culture. Things which I might not have thought about beforehand.

\subsection{Recommendations}

Everyone makes transitions in their lives. Sometimes good, sometimes bad, all different spectrums of emotion can be present. It's interesting to learn about transitions that have been made by some. It's also valuable to inter-personal interest, because it can help two groups share common roots in some cases. I'm curious, I have no doubt I'll continue asking about cultural transitions I find in people, or taking part in them myself.

\section{Chapter 9: Popular Culture}

\subsection{Tactic}

We can learn and gain some experience about other cultures based on popular culture understanding (``Martin, and Nakayama'' 358). By attempting to find some common interests with new acquaintances or existing friends that relate to pop-culture or current events of some kind, I can gain new understanding of cultures and peoples identities. If unable to find apparent pop-culture interests bring up some pop-culture topics.

\subsection{Implementation}

I found topics of conversation that relate to pop-culture that came up naturally in conversation, and while conversing I tried to recognize what kind of communication was going on. Additionally I tried to think about the reasons I focused on certain topics of pop-culture, what it means to be current in pop-culture understanding, and how easy it can be to get wrapped up in trivial nonsense when surrounded with too much of what pop-culture has to offer.

\subsection{Recommendations}

I can't say I'd recommend maintaining interest in pop-culture just to stay relevant to people around you. More like anyone will end up familiar with certain pop-culture events, or information, and it can be a nice topic at times to converse on with people you know, or people you don't actually know and with whom you must simply make small talk. I'll continue to pay attention to the pop-culture interests I have.

\section{Chapter 10: Relationships}


\subsection{Tactic}

The key to being involved in intercultural relationships often involves maintaining balance between differences and similarities (``Martin, and Nakayama'' 391). I hope to pay attention to the relationships I have, whatever type they might be, recognize where cultural differences are to be found, and to ensure that I understand how I can improve my interpersonal interest in these relationships.

\subsection{Implementation}

Through reflection, and further analysis of current relationships I have amongst various cultures. I find myself very different from many different groups around me, I also find myself embedded in other cultures. The only necessary action is to continue those that are positive, and to diminish those relationships that do not benefit to my well-being, or that of those around me.

\subsection{Recommendations}

I will always continue to improve various relationships around me. It would be wrong not to do so. I will always have to take time to analyze what I value in friendships, in coworkers, and in romantic interests. I'll have to consistently re-evaluate what I need in these relationships, and will always have to use my best judgement.

\section{Chapter 11: Conflict}

\subsection{Tactic}

Conflicts are inevitable, happening all around the world, and at many different levels (``Martin, and Nakayama'' 435). I hope to recognize apparent, but ambiguous points of conflict in my life currently, and analyze my reasoning and positions on them. Should I be more considerate? Should I focus on improving communication? Should I seek further action? Should I admit defeat? Most importantly, in-line with my goal, is this an issue of not taking personal interest in the other parties involved in this conflict?

\subsection{Implementation}

I deal with a decent amount of regular conflict sometimes with those around me. There is no more conflicting time than forcing a bunch of students to work on comprehensive projects across multiple topics in multiple classes with a rushed deadline, and with people who may or may not be as interested in the material as you. I think it's a good idea to review the conflict occurring in the final projects I'm working on, and try to understand what toll they're taking.

\subsection{Recommendations}

This is a hard one. Conflict can be tricky to deal with in a clean fashion, and I respect those who can do it regularly and repeatably. I think I'll continue to write about conflicts that I find myself in, though I will rarely share it. It's therapeutic, and can help materialize the type of message and communication you need in order to deal with conflict as an adult.

\section{Reflection}

I've enjoyed reflecting on various acts of communication over the course of the semester. It's easy to take certain things for granted, until you require yourself to take a moment and observe what it is you're doing. I feel that parts of the goal which I had chosen may not have always applied to my situation, and that lead to some difficulty in contriving some of the thoughts I had in my journals. Overall it was a beneficial project, allowing me to examine and improve certain characteristics in life that could always use improvement.

\pagebreak
\section{Works Cited}

\bibent
GCI, Communication 2150 Course Readings. SLCC, Spring 2013.
\bibent

Martin, Judith N., and Thomas K. Nakayama. ``Chapter 10: Culture,
Communication, and Intercultural Relationships.''
\textit{Intercultural Communication in Contexts.} 6th ed. Boston: 
McGraw-Hill, 2013. 391. Print.


\end{mla}
\end{document}

