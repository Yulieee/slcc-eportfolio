\documentclass[12pt]{article}

\usepackage{graphicx}

\pagestyle{empty}
\setcounter{secnumdepth}{2}

\topmargin=0cm
\oddsidemargin=0cm
\textheight=22.0cm
\textwidth=16cm
\parindent=0cm
\parskip=0.15cm
\topskip=0truecm
\raggedbottom
\abovedisplayskip=3mm
\belowdisplayskip=3mm
\abovedisplayshortskip=0mm
\belowdisplayshortskip=2mm
\normalbaselineskip=12pt
\normalbaselines

\begin{document}

\begin{table}[htbp]
\caption{Instantiation of a \textbf{Distance} object using default values for the attributes.}
\begin{center}
\noindent\resizebox{\textwidth}{!}{%
\begin{tabular}{ | l | p{5cm} | p{5cm} | p{5cm} | }
\hline
Operation & Purpose & Object State & Expected Result \\
\hline\hline

% Row
Distance d = new Distance() & 
To create a distance using the default values. & 
\textbf{feet = 1} \newline \textbf{inches = 1} & 
A new Distance object with default values for the attributes. \\
\hline

% Row
d.getFeet() &
To verify instantiation and accessor method. &
&
1 \\
\hline

% Row
d.getInches() &
To verify instantiation and accessor method. &
&
1 \\
\hline


\end{tabular}
}
\end{center}
\end{table}


%%%%%%%%%%%%

\begin{table}[htbp]
\caption{Instantiation of a \textbf{Distance} object with legal, client-supplied values for the attributes.}
\begin{center}
\noindent\resizebox{\textwidth}{!}{%
\begin{tabular}{ | l | p{5cm} | p{5cm} | p{5cm} | }
\hline
Operation & Purpose & Object State & Expected Result \\
\hline\hline

% Row
Distance d2 = new Distance(3, 5) &
To create a distance using constructor values. &
\textbf{feet = 3} \newline \textbf{inches = 5} &
A new Distance object with constructed values for the attributes. \\
\hline



% Row
d.getFeet() &
To verify instantiation and accessor method. &
&
3 \\
\hline

% Row
d.getInches() &
To verify instantiation and accessor method. &
&
5 \\
\hline


\end{tabular}
}
\end{center}
\end{table}


%%%%%%%%%%%%

\begin{table}[htbp]
\caption{Valid functionality of the getter \& setter methods available.}
\begin{center}
\noindent\resizebox{\textwidth}{!}{%
\begin{tabular}{ | l | p{5cm} | p{5cm} | p{5cm} | }
\hline
Operation & Purpose & Object State & Expected Result \\
\hline\hline

% Row
Distance d = new Distance() & 
To create a distance using the default values. & 
\textbf{feet = 1} \newline \textbf{inches = 1} & 
A new Distance object with default values for the attributes. \\
\hline

% Row
d.getFeet() &
To verify instantiation and accessor method. &
&
1 \\
\hline

% Row
d.getInches() &
To verify instantiation and accessor method. &
&
1 \\
\hline

% Row
d.setFeet(3) &
To ensure attributes get set properly. &
\textbf{feet = 3} \newline inches = 1 &
The Distance object contains the new value for feet. \\
\hline

% Row
d.setInches(5) &
To ensure attributes get set properly. &
feet = 3 \newline \textbf{inches = 5} &
The Distance object contains the new value for inches. \\
\hline

% Row
d.setFeet(-2) &
To ensure validation of attributes. &
&
FeetOutOfRangeException \\
\hline

% Row
d.setInches(-2) &
To ensure validation of attributes. &
&
InchesOutOfRangeException \\
\hline

% Row
Distance d2 = new Distance(3, 5) &
To create a distance using constructor values. &
\textbf{feet = 3} \newline \textbf{inches = 5} &
A new Distance object with constructed values for the attributes. \\
\hline

% Row
d2.getFeet() &
To verify instantiation and accessor method. &
&
3 \\
\hline

% Row
d2.getInches() &
To verify instantiation and accessor method. &
&
5 \\
\hline


\end{tabular}
}
\end{center}
\end{table}




%%%%%%%%%%%%

\begin{table}[htbp]
\caption{Valid functionality of the addition and subtraction methods for the \textbf{Distance} object.}
\begin{center}

\noindent\resizebox{\textwidth}{!}{%

\begin{tabular}{ | l | p{5cm} | p{5cm} | p{5cm} | }
\hline
Operation & Purpose & Object State & Expected Result \\
\hline\hline

% Row
Distance d = new Distance() & 
To create a distance using the default values. & 
\textbf{feet = 1} \newline \textbf{inches = 1} & 
A new Distance object with default values for the attributes. \\
\hline

% Row
Distance d2 = new Distance(3, 5) & 
To create a distance using constructed values. & 
\textbf{feet = 3} \newline \textbf{inches = 5} & 
A new Distance object with default values for the attributes. \\
\hline

% Row
Distance d3 = Distance.add(d, d2) &
To ensure two distances can be added together &
\textbf{feet = 4} \newline \textbf{inches = 6} &
A new Distance object with attribute values equivalent to the sum of object properties. \\
\hline

% Row
Distance d4 = Distance.subtract(d2, d) &
To ensure two distances can be subtracted from each other &
\textbf{feet = 2} \newline \textbf{inches = 4} &
A new Distance object with attribute values equivalent to the difference of object properties. \\
\hline


% Row
Distance d4 = Distance.subtract(d, d2) &
To ensure two distances subtracted from each other cannot create an invalid object &
&
FeetOutOfRangeException | InchesOutOfRangeException \\
\hline


\end{tabular}

}


\end{center}

\end{table}








%%%%%%%%%%%%

\begin{table}[htbp]
\caption{Valid functionality of equality \& hash code methods for the \textbf{Distance} object.}
\begin{center}

\noindent\resizebox{\textwidth}{!}{%

\begin{tabular}{ | l | p{5cm} | p{5cm} | p{5cm} | }
\hline
Operation & Purpose & Object State & Expected Result \\
\hline\hline

% Row
Distance d = new Distance() & 
To create a distance using the default values. & 
\textbf{feet = 1} \newline \textbf{height = 1} & 
A new Distance object with default values for the attributes. \\
\hline

% Row
Distance d2 = new Distance() & 
To create a distance using constructed values. & 
\textbf{feet = 3} \newline \textbf{height = 5} & 
A new Distance object with constructed values for the attributes. \\
\hline

% Row
Distance d3 = new Distance() & 
To create a distance using default values. & 
\textbf{feet = 1} \newline \textbf{height = 1} & 
A new Distance object with default values for the attributes. \\
\hline

% Row
d.equals(d2) &
To verify two distances are not equal. &
&
False \\
\hline

% Row
d.equals(d3) &
To verify two distances are equal. &
&
True \\
\hline

% Row
d.hashCode() &
To verify the hash code function. &
&
31285 \\
\hline

% Row
d.hashCode() == d2.hashCode() &
To verify two hash codes do not match. &
&
False \\
\hline

% Row
d.hashCode() == d3.hashCode() &
To verify two equal objects have hash codes that match. &
&
True \\
\hline


\end{tabular}

}


\end{center}

\end{table}








%%%%%%%%%%%%

\begin{table}[htbp]
\caption{Valid functionality of the comparability of the \textbf{Distance} object.}
\begin{center}

\noindent\resizebox{\textwidth}{!}{%

\begin{tabular}{ | l | p{5cm} | p{5cm} | p{5cm} | }
\hline
Operation & Purpose & Object State & Expected Result \\
\hline\hline

% Row
Distance d = new Distance() & 
To create a distance using the default values. & 
\textbf{feet = 1} \newline \textbf{inches = 1} & 
A new Distance object with default values for the attributes. \\
\hline

% Row
Distance d2 = new Distance() & 
To create a distance using constructed values. & 
\textbf{feet = 3} \newline \textbf{height = 5} & 
A new Distance object with constructed values for the attributes. \\
\hline

% Row
Distance d3 = new Distance() & 
To create a distance using default values. & 
\textbf{feet = 1} \newline \textbf{height = 1} & 
A new Distance object with default values for the attributes. \\
\hline

% Row
d.compareTo(d2) &
To verify a distance compares less than another. &
&
-1 \\
\hline

% Row
d.compareTo(d3) &
To verify a distance compares the same as another. &
&
0 \\
\hline

% Row
d2.compareTo(d) &
To verify a distance compares greater than another. &
&
1 \\
\hline


\end{tabular}

}


\end{center}

\end{table}


















%%%%%%%%%%%%%%




\end{document}
