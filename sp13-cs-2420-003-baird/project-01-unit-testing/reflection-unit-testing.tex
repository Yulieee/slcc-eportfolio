\documentclass[12pt,letterpaper]{article}
\usepackage{mla}
\usepackage{wrapfig}

\begin{document}
\begin{mla}{Paul}{English}{CS 2420-003}{Robert Baird}{\today}{\textbf{Reflection - Unit Testing}}

Sometimes it's difficult to actually drive development with thorough
testing. Actually, it's always difficult, I've found I rarely get an
adequate specification of what I'm attempting to build, let alone a
test plan. It usually falls on my own hands to take care of the test
plan, and setting up test cases that cover edge cases. I've never
written up a formal test plan, though I have gotten relatively
stalwart about keeping tests up to date.

The more you practice testing your code at the unit level, oftentimes
the better you get at architecting as well. To make a project
testable, it usually needs some level of loose coupling that allows
you to observe the end-to-end interactions of a component or class.
Java does help in providing obvious points of test integration. Other
languages may not be as organized, though they are all testable.

The process of writing code requires a developer to continuely run
ad-hoc tests, so formal/automated tests are really just a way of
saving work, and allowing it to be played back at a later date, once
you feel like augmenting things. It's indespensible in any non-trivial
project and the only way to guarantee observed bugs are fixed
long-term.

Tests don't just need to be relegated to the realm of valid function
responses and ensuring the right exception is or isn't thrown. A
mature test suite may guarantee the efficiency of certain important
functionality, or it might even be able to simulate a user with
high-level interaction using some kind of integration test.

Working with JUnit feels like Java, it is Java, so I can't say I have
much to comment about it. I tend to prefer concise and terse
programming platforms, especially when it comes to testing. With that
said, if it works, it works.

\end{mla}
\end{document}

