\documentclass[12pt,letterpaper]{article}
\usepackage{changepage}
\usepackage{listliketab}
\usepackage{graphicx}


\title{The Hash Table ADT}
\date{\today}
\author{Paul English}

\topmargin=0cm
\oddsidemargin=0cm
\textheight=22.0cm
\textwidth=16cm
\parindent=0cm
\parskip=0.15cm
\topskip=0truecm
\raggedbottom
\abovedisplayskip=3mm
\belowdisplayskip=3mm
\abovedisplayshortskip=0mm
\belowdisplayshortskip=2mm
\normalbaselineskip=12pt
\normalbaselines



\begin{document}

\maketitle

\section*{Description}

A hash table provides a collection that utilizes a hash function to enable O(1) access time to items in the collection.

\section*{Properties}

\begin{enumerate}

\item Duplicate keys are not allowed.
\item A key may be associated with only one value.
\item A value may be associated with more than one key.
\item Keys can be compared to one another for equality; similarly for values.
\item Null keys and values are not allowed.
\item Keys can be hashed into unique values.
\item Handles collisions using a repeatable strategy, if two keys produce the same hash.

\end{enumerate}

\section*{Attributes}

\begin{description}
\item[buckets:] The number of buckets used to store items in.
\item[load factor:] The threshold at which a hash table will rebalance the elements in it's buckets.
\end{description}

\section*{Operations}

% TODO
\textbf{HashTable()} \\
\begin{listliketab} 
    \storestyleof{itemize} 
        \begin{tabular}{Lll}
            & pre-condition:    & none \\
            & responsibilities: & constructor - create an empty hash map \\
            & post-condition:   & \emph{size} is set to 0 \\
            & return:           & nothing
        \end{tabular} 
\end{listliketab}
    
\textbf{put( KeyType key, ValueType value )} \\
\begin{listliketab} 
    \storestyleof{itemize} 
        \begin{tabular}{Lll}
            & pre-condition:    & none \\
            & responsibilities: & constructor - create an empty hash map \\
            & post-condition:   & \emph{size} is set to 0 \\
            & return:           & nothing
        \end{tabular} 
    \end{listliketab}
    
\textbf{get( KeyType key )} \\
\begin{listliketab} 
    \storestyleof{itemize} 
        \begin{tabular}{Lll}
            & pre-condition:    & none \\
            & responsibilities: & constructor - create an empty hash map \\
            & post-condition:   & \emph{size} is set to 0 \\
            & return:           & nothing
        \end{tabular} 
    \end{listliketab}
    
\textbf{remove( KeyType key )} \\
\begin{listliketab} 
    \storestyleof{itemize} 
        \begin{tabular}{Lll}
            & pre-condition:    & none \\
            & responsibilities: & constructor - create an empty hash map \\
            & post-condition:   & \emph{size} is set to 0 \\
            & return:           & nothing
        \end{tabular} 
    \end{listliketab}
    
\textbf{containsValue( ValueType value )} \\
\begin{listliketab} 
    \storestyleof{itemize} 
        \begin{tabular}{Lll}
            & pre-condition:    & none \\
            & responsibilities: & constructor - create an empty hash map \\
            & post-condition:   & \emph{size} is set to 0 \\
            & return:           & nothing
        \end{tabular}
    \end{listliketab}
    
\textbf{containsKey( KeyType key )} \\
\begin{listliketab} 
    \storestyleof{itemize} 
        \begin{tabular}{Lll}
            & pre-condition:    & none \\
            & responsibilities: & constructor - create an empty hash map \\
            & post-condition:   & \emph{size} is set to 0 \\
            & return:           & nothing
        \end{tabular} 
    \end{listliketab}
    
\textbf{values()} \\
\begin{listliketab} 
    \storestyleof{itemize} 
        \begin{tabular}{Lll}
            & pre-condition:    & none \\
            & responsibilities: & constructor - create an empty hash map \\
            & post-condition:   & \emph{size} is set to 0 \\
            & return:           & nothing
        \end{tabular} 
    \end{listliketab}
    
% TODO hash function
% TODO insertions
% TODO deletion
% TODO searches
% TODO values
    

\textbf{Not shown:} clear(), isEmpty(), and size()

\section*{Test Plan}

% TODO

\begin{table}[htbp]
\caption{Instantiation of a \textbf{Distance} object using default values for the attributes.}
\begin{center}
\noindent\resizebox{\textwidth}{!}{%
\begin{tabular}{ | l | p{5cm} | p{5cm} | p{5cm} | }
\hline
Operation & Purpose & Object State & Expected Result \\
\hline\hline

% Row
Distance d = new Distance() & 
To create a distance using the default values. & 
\textbf{feet = 1} \newline \textbf{inches = 1} & 
A new Distance object with default values for the attributes. \\
\hline

% Row
d.getFeet() &
To verify instantiation and accessor method. &
&
1 \\
\hline

% Row
d.getInches() &
To verify instantiation and accessor method. &
&
1 \\
\hline


\end{tabular}
}
\end{center}
\end{table}


%%%%%%%%%%%%


\end{document}
