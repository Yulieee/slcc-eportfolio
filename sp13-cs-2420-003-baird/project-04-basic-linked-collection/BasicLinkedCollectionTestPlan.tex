\documentclass[12pt]{article}

\usepackage{graphicx}

\pagestyle{empty}
\setcounter{secnumdepth}{2}

\topmargin=0cm
\oddsidemargin=0cm
\textheight=22.0cm
\textwidth=16cm
\parindent=0cm
\parskip=0.15cm
\topskip=0truecm
\raggedbottom
\abovedisplayskip=3mm
\belowdisplayskip=3mm
\abovedisplayshortskip=0mm
\belowdisplayshortskip=2mm
\normalbaselineskip=12pt
\normalbaselines

\begin{document}


%%%%%%%%%%%%%%%%%%%%%%%%%%%%%%%%%%%%%%%%%%%%%%%%%%%%%%%%%%%%%%%%%%%%%%%
\begin{table}[htbp]
\caption{Instantiation of a \textbf{BasicCollection}}
\begin{center}

\noindent\resizebox{\textwidth}{!}{%

\begin{tabular}{ | l | p{5cm} | p{5cm} | p{5cm} | }
\hline
Operation & Purpose & Object State & Expected Result \\
\hline\hline

% Row
Collection c = new BasicCollection() & 
To create a basic collection with the default constructor. & 
\textbf{size = 0} \newline \textbf{empty = true} & 
A new Collection object with default values for the attributes. \\
\hline

% Row
c.size() &
To verify instantiatied size. &
&
0 \\
\hline

% Row
c.isEmpty() &
To verify instantiated object is empty. &
&
true \\
\hline

\end{tabular}

}
\end{center}
\end{table}

%%%%%%%%%%%%%%%%%%%%%%%%%%%%%%%%%%%%%%%%%%%%%%%%%%%%%%%%%%%%%%%%%%%%%%%
\begin{table}[htbp]
\caption{Adding objects to an instantiated collection.}
\begin{center}

\noindent\resizebox{\textwidth}{!}{%

\begin{tabular}{ | l | p{5cm} | p{5cm} | p{5cm} | }
\hline
Operation & Purpose & Object State & Expected Result \\
\hline\hline

% Row
BasicCollection $<$String$>$ c = new BasicCollection$<$String$>$() & 
To create a basic string collection with the default constructor. & 
\textbf{size = 0} \newline \textbf{empty = true} & 
A new Collection object with default values for the attributes. \\
\hline

% Row
c.add(new String("A")) &
Add a string to the collection. &
\textbf{size = 1} \newline \textbf{empty = false} &
The collection contains the string we've added.\\
\hline

% Row
!c.isEmpty() &
To verify instantiated object isn't empty. &
&
true \\
\hline

%Row
c.contains("A") &
Check that our collection contains the string added. &
&
true \\
\hline

%Row
c.contains("Missing") &
To verify that other objects don't falsely report to be in our collection. &
&
false \\
\hline

\end{tabular}

}
\end{center}
\end{table}

%%%%%%%%%%%%%%%%%%%%%%%%%%%%%%%%%%%%%%%%%%%%%%%%%%%%%%%%%%%%%%%%%%%%%%%
\begin{table}[htbp]
\caption{Ability to remove unique elements from the collection.}
\begin{center}

\noindent\resizebox{\textwidth}{!}{%

\begin{tabular}{ | l | p{5cm} | p{5cm} | p{5cm} | }
\hline
Operation & Purpose & Object State & Expected Result \\
\hline\hline

% Row
BasicCollection $<$String$>$ c = new BasicCollection$<$String$>$() & 
To create a basic string collection with the default constructor. & 
\textbf{size = 0} \newline \textbf{empty = true} & 
A new Collection object with default values for the attributes. \\
\hline

% Row
c.add(new String("A")) &
Add a string to the collection. &
\textbf{size = 1} \newline \textbf{empty = false} &
The collection contains the string we've added.\\
\hline

% Row
c.add(new String("B")) &
Add another string to the collection. &
\textbf{size = 2} \newline \textbf{empty = false} &
The collection contains the string we've added.\\
\hline

% Row
!c.isEmpty() &
To verify instantiated object isn't empty. &
&
true \\
\hline

%Row
c.size() &
Check that our contains two elements. &
&
2 \\
\hline

%Row
c.remove("B") &
To remove an individual item from the collection. &
\textbf{size = 1} &
\\
\hline

%Row
c.size() &
To check the updated size of our collection. &
&
1 \\
\hline

%Row
c.contains("B") &
To ensure our item was removed. &
&
false \\
\hline

\end{tabular}

}
\end{center}
\end{table}


%%%%%%%%%%%%%%%%%%%%%%%%%%%%%%%%%%%%%%%%%%%%%%%%%%%%%%%%%%%%%%%%%%%%%%%
\begin{table}[htbp]
\caption{Ability to add multiple items to a \textbf{BasicCollection}.}
\begin{center}

\noindent\resizebox{\textwidth}{!}{%

\begin{tabular}{ | l | p{5cm} | p{5cm} | p{5cm} | }
\hline
Operation & Purpose & Object State & Expected Result \\
\hline\hline

% Row
ArrayList$<$String$>$ list = new ArrayList$<>$(); &
Create a random collection that can be used to add to our collection. &
&
\\
\hline

% Row
list.add("A"); &
Add an element to our collection. &
&
\\
\hline

% Row
list.add("B"); &
Add an element to our collection. &
&
\\
\hline

% Row
list.add("C"); &
Add an element to our collection. &
&
\\
\hline

% Row
c = new BasicCollection$<>$(); &
Create a basic collection using the default constructor. &
\textbf{size = 0} \newline \textbf{empty = true} &
A new collection object. \\
\hline

% Row
c.addAll(list); &
Add our list to our collection. &
\textbf{size = 3} \newline \textbf{empty = false} &
A populated collection object. \\
\hline

% Row
c.contains("A"); &
Make sure our collection contains this element that was in the list. &
&
true \\
\hline

% Row
c.contains("B"); &
Make sure our collection contains this element that was in the list. &
&
true \\
\hline

% Row
c.contains("C"); &
Make sure our collection contains this element that was in the list. &
&
true \\
\hline

% Row
c.size() &
Make sure our collection has the right sizes. &
&
3 \\
\hline

\end{tabular}

}
\end{center}
\end{table}


%%%%%%%%%%%%%%%%%%%%%%%%%%%%%%%%%%%%%%%%%%%%%%%%%%%%%%%%%%%%%%%%%%%%%%%
\begin{table}[htbp]
\caption{Ability to remove all items to a \textbf{BasicCollection}.}
\begin{center}

\noindent\resizebox{\textwidth}{!}{%

\begin{tabular}{ | l | p{5cm} | p{5cm} | p{5cm} | }
\hline
Operation & Purpose & Object State & Expected Result \\
\hline\hline

% Row
ArrayList$<$String$>$ list = new ArrayList$<>$(); &
Create a random collection that can be used to add to our collection. &
&
\\
\hline

% Row
list.add("A"); &
Add an item to our list. &
&
\\
\hline

% Row
list.add("B"); &
Add an item to our list. &
&
\\
\hline

% Row
list.add("C"); &
Add an item to our list. &
&
\\
\hline

% Row
c = new BasicCollection$<>$(); &
Create a basic collection using the default constructor. &
\textbf{size = 0} \newline \textbf{empty = true} &
A new collection object. \\
\hline

% Row
c.add("A"); &
Add an item to our collection. &
\textbf{size = 1} \newline \textbf{empty = false} &
Our collection. \\
\hline

% Row
c.add("B"); &
Add an item to our collection. &
\textbf{size = 2} &
Our collection. \\
\hline

% Row
c.add("C"); &
Add an item to our collection. &
\textbf{size = 3} &
Our collection. \\
\hline

% Row
c.removeAll(list); &
Remove all list elements from our collection. &
\textbf{size = 0} \newline \textbf{empty = true} &
Our collection has had all the elements removed leaving it empty. \\
\hline

% Row
c.contains("A"); &
Make sure our collection doesn't have an element in it. &
&
false \\
\hline

% Row
c.contains("B"); &
Make sure our collection doesn't have an element in it. &
&
false \\
\hline

% Row
c.contains("C"); &
Make sure our collection doesn't have an element in it. &
&
false \\
\hline

% Row
c.size(); &
Make sure our collection now has the correct size. &
&
0 \\
\hline

% Row
c.isEmpty(); &
Make sure our collection is empty. &
&
true \\
\hline

\end{tabular}

}
\end{center}
\end{table}


%%%%%%%%%%%%%%%%%%%%%%%%%%%%%%%%%%%%%%%%%%%%%%%%%%%%%%%%%%%%%%%%%%%%%%%
\begin{table}[htbp]
\caption{When removing all items from a \textbf{BasicCollection} don't remove other elements.}
\begin{center}

\noindent\resizebox{\textwidth}{!}{%

\begin{tabular}{ | l | p{5cm} | p{5cm} | p{5cm} | }
\hline
Operation & Purpose & Object State & Expected Result \\
\hline\hline

% Row
ArrayList$<$String$>$ list = new ArrayList$<>$(); &
Create a random collection that can be used to add to our collection. &
&
\\
\hline

% Row
list.add("A"); &
Add an item to our list. &
&
\\
\hline

% Row
list.add("B"); &
Add an item to our list. &
&
\\
\hline

% Row
list.add("C"); &
Add an item to our list. &
&
\\
\hline

% Row
c = new BasicCollection$<>$(); &
Create a basic collection using the default constructor. &
\textbf{size = 0} \newline \textbf{empty = true} &
A new collection object. \\
\hline

% Row
c.add("A"); &
Add an element to our collection. &
\textbf{size = 1} \newline \textbf{empty = false} &
Our collection with one element. \\
\hline

% Row
c.add("D"); &
Add an element to our collection. &
\textbf{size = 2} &
Our collection with two elements. \\
\hline

% Row
c.removeAll(list); &
Remove the list items from our collection again. &
\textbf{size = 1} &
Our collection still has one element in it. \\
\hline

% Row
c.contains("A"); &
Check that our element doesn't contain the list element. &
&
false \\
\hline

% Row
c.contains("D"); &
Check that our element still contains the other element. &
&
true \\
\hline

% Row
c.size(); &
Check that our collection has the right size. &
&
1 \\
\hline

% Row
c.isEmpty(); &
Check that our element isn't empty. &
&
false \\
\hline

% Row
BasicCollection $<$String$>$ c = new BasicCollection$<>$() & 
To create a basic string collection with the default constructor. & 
\textbf{size = 0} \newline \textbf{empty = true} & 
A new Collection object with default values for the attributes. \\
\hline

% Row
c.add(new String("A")) &
Add a string to the collection. &
\textbf{size = 1} \newline \textbf{empty = false} &
The collection contains the string we've added.\\
\hline

% Row
!c.isEmpty() &
To verify instantiated object isn't empty. &
&
true \\
\hline

%Row
c.contains("A") &
Check that our collection contains the string added. &
&
true \\
\hline

%Row
c.contains("Missing") &
To verify that other objects don't falsely report to be in our collection. &
&
false \\
\hline

\end{tabular}

}
\end{center}
\end{table}

\end{document}
