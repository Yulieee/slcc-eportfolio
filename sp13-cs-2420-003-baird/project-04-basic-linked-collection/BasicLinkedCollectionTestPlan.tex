\documentclass[12pt]{article}

\usepackage{graphicx}

\pagestyle{empty}
\setcounter{secnumdepth}{2}

\topmargin=0cm
\oddsidemargin=0cm
\textheight=22.0cm
\textwidth=16cm
\parindent=0cm
\parskip=0.15cm
\topskip=0truecm
\raggedbottom
\abovedisplayskip=3mm
\belowdisplayskip=3mm
\abovedisplayshortskip=0mm
\belowdisplayshortskip=2mm
\normalbaselineskip=12pt
\normalbaselines

\begin{document}


%%%%%%%%%%%%%%%%%%%%%%%%%%%%%%%%%%%%%%%%%%%%%%%%%%%%%%%%%%%%%%%%%%%%%%%
\begin{table}[htbp]
\caption{Instantiation of a \textbf{BasicCollection}}
\begin{center}

\noindent\resizebox{\textwidth}{!}{%

\begin{tabular}{ | p{5cm} | p{5cm} | p{5cm} | p{5cm} | }
\hline
Operation & Purpose & Object State & Expected Result \\
\hline\hline

% Row
Collection c = new BasicCollection() & 
To create a basic collection with the default constructor. & 
\textbf{size = 0} \newline \textbf{empty = true} & 
A new Collection object with default values for the attributes. \\
\hline

% Row
c.size() &
To verify instantiatied size. &
&
0 \\
\hline

% Row
c.isEmpty() &
To verify instantiated object is empty. &
&
true \\
\hline

\end{tabular}

}
\end{center}
\end{table}

%%%%%%%%%%%%%%%%%%%%%%%%%%%%%%%%%%%%%%%%%%%%%%%%%%%%%%%%%%%%%%%%%%%%%%%
\begin{table}[htbp]
\caption{Adding objects to an instantiated collection.}
\begin{center}

\noindent\resizebox{\textwidth}{!}{%

\begin{tabular}{ | p{5cm} | p{5cm} | p{5cm} | p{5cm} | }
\hline
Operation & Purpose & Object State & Expected Result \\
\hline\hline

% Row
BasicCollection<String> c = new BasicCollection<String>() & 
To create a basic string collection with the default constructor. & 
\textbf{size = 0} \newline \textbf{empty = true} & 
A new Collection object with default values for the attributes. \\
\hline

% Row
c.add(new String("A")) &
Add a string to the collection. &
\textbf{size = 1} \newline \textbf{empty = false} &
The collection contains the string we've added.\\
\hline

% Row
!c.isEmpty() &
To verify instantiated object isn't empty. &
&
true \\
\hline

%Row
c.contains("A") &
Check that our collection contains the string added. &
&
true \\
\hline

%Row
c.contains("Missing") &
To verify that other objects don't falsely report to be in our collection. &
&
false \\
\hline

\end{tabular}

}
\end{center}
\end{table}

%%%%%%%%%%%%%%%%%%%%%%%%%%%%%%%%%%%%%%%%%%%%%%%%%%%%%%%%%%%%%%%%%%%%%%%
\begin{table}[htbp]
\caption{Ability to remove unique elements from the collection.}
\begin{center}

\noindent\resizebox{\textwidth}{!}{%

\begin{tabular}{ | p{5cm} | p{5cm} | p{5cm} | p{5cm} | }
\hline
Operation & Purpose & Object State & Expected Result \\
\hline\hline

% Row
BasicCollection<String> c = new BasicCollection<String>() & 
To create a basic string collection with the default constructor. & 
\textbf{size = 0} \newline \textbf{empty = true} & 
A new Collection object with default values for the attributes. \\
\hline

% Row
c.add(new String("A")) &
Add a string to the collection. &
\textbf{size = 1} \newline \textbf{empty = false} &
The collection contains the string we've added.\\
\hline

% Row
c.add(new String("B")) &
Add another string to the collection. &
\textbf{size = 2} \newline \textbf{empty = false} &
The collection contains the string we've added.\\
\hline

% Row
!c.isEmpty() &
To verify instantiated object isn't empty. &
&
true \\
\hline

%Row
c.size() &
Check that our contains two elements. &
&
2 \\
\hline

%Row
c.remove("B") &
To remove an individual item from the collection. &
\textbf{size = 1} &
\\
\hline

%Row
c.size() &
To check the updated size of our collection. &
&
1 \\
\hline

%Row
c.contains("B") &
To ensure our item was removed. &
&
false \\
\hline

\end{tabular}

}
\end{center}
\end{table}


%%%%%%%%%%%%%%%%%%%%%%%%%%%%%%%%%%%%%%%%%%%%%%%%%%%%%%%%%%%%%%%%%%%%%%%
\begin{table}[htbp]
\caption{Ability to add multiple items to a \textbf{BasicCollection}.}
\begin{center}

\noindent\resizebox{\textwidth}{!}{%

\begin{tabular}{ | p{5cm} | p{5cm} | p{5cm} | p{5cm} | }
\hline
Operation & Purpose & Object State & Expected Result \\
\hline\hline

% Row
BasicCollection<String> c = new BasicCollection<String>() & 
To create a basic string collection with the default constructor. & 
\textbf{size = 0} \newline \textbf{empty = true} & 
A new Collection object with default values for the attributes. \\
\hline

% Row
c.add(new String("A")) &
Add a string to the collection. &
\textbf{size = 1} \newline \textbf{empty = false} &
The collection contains the string we've added.\\
\hline

% Row
!c.isEmpty() &
To verify instantiated object isn't empty. &
&
true \\
\hline

%Row
c.contains("A") &
Check that our collection contains the string added. &
&
true \\
\hline

%Row
c.contains("Missing") &
To verify that other objects don't falsely report to be in our collection. &
&
false \\
\hline

\end{tabular}

}
\end{center}
\end{table}


%%%%%%%%%%%%%%%%%%%%%%%%%%%%%%%%%%%%%%%%%%%%%%%%%%%%%%%%%%%%%%%%%%%%%%%
\begin{table}[htbp]
\caption{Ability to remove all items to a \textbf{BasicCollection}.}
\begin{center}

\noindent\resizebox{\textwidth}{!}{%

\begin{tabular}{ | p{5cm} | p{5cm} | p{5cm} | p{5cm} | }
\hline
Operation & Purpose & Object State & Expected Result \\
\hline\hline

% Row
BasicCollection<String> c = new BasicCollection<String>() & 
To create a basic string collection with the default constructor. & 
\textbf{size = 0} \newline \textbf{empty = true} & 
A new Collection object with default values for the attributes. \\
\hline

% Row
c.add(new String("A")) &
Add a string to the collection. &
\textbf{size = 1} \newline \textbf{empty = false} &
The collection contains the string we've added.\\
\hline

% Row
!c.isEmpty() &
To verify instantiated object isn't empty. &
&
true \\
\hline

%Row
c.contains("A") &
Check that our collection contains the string added. &
&
true \\
\hline

%Row
c.contains("Missing") &
To verify that other objects don't falsely report to be in our collection. &
&
false \\
\hline

\end{tabular}

}
\end{center}
\end{table}

\end{document}
