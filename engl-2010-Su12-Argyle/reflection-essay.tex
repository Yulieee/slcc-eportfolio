\documentclass[12pt,letterpaper]{article}
\usepackage{mla}
\usepackage{wrapfig}

\begin{document}
\begin{mla}{Paul}{English}{ENG 2010-002}{Professor Argyle}{\today}{\textbf{Reflection Essay}}

In revising, writing, and working through this course I both learned greatly and improved my individual writing style and voice. In observing reviews and comments to my own papers I found that understanding and fixing grammar and spelling mistakes were useful, but comments on the theme, structure, and flow of my papers was by far the most beneficial.

In working with my peers, I think I've learned the most about focused their individual writing styles. In the past I've often only followed simplistic structures, or just written freely without care of my purpose or goals. Understanding writing structure for different classifications of essay has certainly helped to metaphorically expand my language arsenal.

As I've been reviewing with my peers, my suggestions most often referenced their content. Though I did note grammatical errors, or spelling mistakes, I was always most interested in what they were writing about and why. It shouldn't be too shocking therefore that my corrections or suggestions often centered around the writers topic. I often asked many questions about the theme of the papers I reviewed.

I suppose in a way it's easy for me to find interest whatever it was my team and other group members were writing. I find it's hard to provide quality feedback if you don't really care too much about what you have recently read.

Reviewing has helped me to expand in my own writing. I'm more able to read and decipher the voice that authors have, and I find that characteristics that I value in their style may have begun showing through in my own writing.

As I reviewed my own work, and revised for my final portfolio there were many things I changed and focused on. Re-reading my papers I tried to elaborate on difficult points of understanding, and attempted to reduce passive voice that I overlooked in my drafts. I also found that some of the automatic spell check features included in my operating system tended to augment things I wrote in unexpected ways. It was fairly common for me to find correctly spelled homonyms which needed to be changed to preserve the meaning I had initially intended. Even in writing this reflection essay, I found my editor aggressively fixing words I typed. Needless to say this OS setting has been properly adjusted.

I chose the pieces I enjoyed writing the most. Largely due to their topic, and the effort that I put in to researching them. I hope that one of the greater strengths of my portfolio is an unbiased view of data, events, and people. Though we all suffer from our own blinders when it comes to this, and additional peer review could likely help me to fix some of these issues where they are found.

If given the chance to do it again, I would probably focus a bit more on getting my thoughts out on paper earlier in the process. Hopefully this would allow more thorough peer review.

All in all, I did find much of this course to be satisfying, and I hope to improve on the things I've learned over the coming years.

\end{mla}
\end{document}
