\documentclass[12pt,letterpaper]{article}
\usepackage{mla}
\begin{document}
\begin{mla}{Paul}{English}{Professor Argyle}{Environmental Degradation}{\today}{Annotated Bibliography}

\bibent
author's last name, first name.  ``Paper Title."  \textit{Book Title}.  Date of publication.

\bibent
Heinberg, Richard, and Daniel Lerch. \textit{The Post Carbon Reader: Managing the 21st Century's Sustainability Crises.} Healdsburg, CA: Watershed Media, 2010. Print.

\indent{\bf (Book) Summary:} Several authors who are related to the environment, and conservation efforts take the reader through a sprawling tour of climate, biology, behavior, economy, resources, and more in order to raise awareness, and provide insight into individual as well as organized contributions to our global environment. We are provided with expert opinions on a variety of topics closely relating to the impact our current rate of consumption, and growth on our planet.

\indent{\bf Assessment:} Each essay has been collected and ordered to target specific issues while providing an over-arching narrative that promotes the transition away from fossil fuels, and towards a proposed cleaner and healthier future. There is a lot of discussion on psychological and political biases that cause us to be unfit at naturally managing our environment without directed action. Many of these issues relate directly to my current topic to inform readers, and spur small changes despite common misunderstandings related to environmental change.

\bibent
J�rgensen, Michael S�gaard, and Ulrik J�rgensen. "Green Technology Foresight Of High Technology: A Social Shaping Of Technology Approach To The Analysis Of Hopes And Hypes." Technology Analysis \& Strategic Management 21.3 (2009): 363-379. Academic Search Premier. Web. 19 June 2012.

\indent{\bf (Scholarly Article) Summary:} This article reviews the practice and benefits that social shaping of technology can lead to in innovation and research. The author makes a point to show that technology foresight use usually linear and falls short of what it could be due to our traditional fallacies in working with new technologies, and offers several methods of improving technological foresight in order to positively impact environmental protection.

\indent{\bf Assessment:} Understanding our own human biases and limited linear imaginations, we are shown that using a few simple techniques we can better innovate, and apply multiple new technologies together in tandem to overshoot our goals in protecting and improving the environment. There is usually more than one way to engineer a solution to a problem, and likewise there are usually more than one application for any given solution. The importance of horizontally thinking in green technologies may be a valuable tool in incrementally improving our global consumption, and mitigating environmental harm.

\bibent
Levi, Maurice D., and Barrie R. Nault. "Converting Technology to Mitigate Environmental Damage." Management Science 50.8 (2004): 1015-030. Print.

\indent{\bf (Scholarly Article) Summary:} Factories and production facilities often have policy, budgets, and output requirements that can make adoption of cleaner processes and tools a difficult barrier to overcome. By examining not only direct costs, but also factoring in long-term efficiency improvements and environmental damage, strategic organizations can improve their technologies quicker and with less political clout.

\indent{\bf Assessment:} Politics, logistics, and general work are all very good reasons for avoiding an update to internal infrastructure. Whether it�s a large factory that uses antiquated machining tools, or a high tech software business, the assets inside can make a huge impact on the environment. Usually upgrading and improving is a difficult process, but having a reason to upgrade that boils down to dollars saved, while improving environmental impact is a win-win situation for most organizations. Reviewing an example of a successful technology conversion in order hamper damage to the environment, makes for a key point in the technology/environment race, and can be leveraged for greater reader understanding.

\bibent
McManus, M.C. "Environmental Consequences Of The Use Of Batteries In Low Carbon Systems: The Impact Of Battery Production." Applied Energy 93. (2012): 288-295. Academic Search Premier. Web. 21 June 2012.

\indent{\bf (Article) Summary:} Batteries are necessary to power millions of day to day services from generators and energy storage systems, to handheld devices and toys. With the current growing trends of technology, we are bound to see an increased use of batteries in many different applications, including things like hybrid or electric vehicles. Though they are useful tool for efficiently working with new technologies, batteries can have a steep negative environmental impact due to the hazardous and scarce materials required to build them. A review of the production and energy impacts is provided for insight and further study.

\indent{\bf Assessment:} Batteries are a common tool that we often overlook in our day-to-day life. Nevertheless, with mobile computing devices, large scale backup and support systems, and even with cars and transportation, it�s nearly impossible not to use a battery in some way every day. Understanding that the lightweight portable devices, and improved gas-mileage found in hybrid/electric vehicles often contains an overlooked cost in producing and sourcing batteries to power our junk. It�s important to recognize the environmental impact behind portable power sources, as well as how the technology is improving.

\bibent
Plous, Evan �Production and Use of Biodegradable Materials for Incorporation in a Non-Toxic, Eco-Friendly Battery� Kansas Academy of Science 110, (2007): 116-124. JSTOR. Web. 21 June 2012.

\indent{\bf (Article) Summary:} Battery use and production has grown greatly over the past few decades, and the use of hazardous, and toxic materials is a bottleneck in providing continued clean energy in a widespread technology. An example eco-friendly battery is reviewed and discussed, in order to prove that this particular piece of technology has a future with clean environments.

\indent{\bf Assessment:} Since the battery has become a core piece of energy transfer and storage, we�ve grown reliant on it as a tool for our modern lifestyle. The overall effect of battery production tends to have a negative impact on the environment, when you account for production, consumption, and hazardous waste. Referencing independent research that shows how a basic battery can be constructed using completely eco-friendly materials provides a glimpse into future directions this technology may take in helping us sustain our environment.

\bibent
UPS. Public Relations. Right Turn at the Right Time. UPS Pressroom. UPS, 2007. Web. 21 June 2012. http://pressroom.ups.com/About+UPS/UPS+Leadership/Speeches/D.+Scott+Davis/Right+Turn+at+the+Right+Time.

\indent{\bf (Press Release) Summary:} Fuel conservation and reputation are important values to UPS as an organization. Before 2007, UPS programmed their mapping and automatic route setting software to avoid left-hand turns if possible that would require idling and increased fuel consumption. In 2007, they reduced CO2 emissions by 32,000 metric tons. They continue to seek out innovative technologies that will help them to curb their environmental impact, as they operate a large distribution fleet.

\indent{\bf Assessment:} Any company that can save 32,000 metric tons of CO2 emissions, simply by instructing, through software, their drivers to take more clockwise routes, and avoid slow left-hand turning lanes, is a company that is working at scale. We tend to dismiss small efficiency and consumption changes we can make in our own lives, simply because the solution isn�t big enough. As a society our small improvements will yield global results, if we work together. Examples of small changes, that provide a large impact, will help to inspire the reader towards action, and understanding.

\bibent
Wal-Mart. Wal-Mart Surpasses Goal To Sell 100 Million Compact Fluorescent Light Bulbs Three Months Early. Wal-Mart Corporate. Wal-Mart, 2 Oct. 2007. Web. 21 June 2012. <http://www.walmartstores.com/pressroom/news/6756.aspx>.

\indent{\bf (Press Release) Summary:} Wal-Mart announces the achievement of its 2007 goal to provide 100 million compact fluorescent light bulbs at an affordable price to consumers. CFL bulbs are more efficient, allowing Wal-Mart to help cut nearly \$3 Billion in consumer utility bills.

\indent{\bf Assessment:} Wal-Mart is a large organization that certainly takes its share in the global environment. When a large company is able to run a campaign targeting specific high efficiency technologies, large environmental impacts can be made quickly and effectively due to the economies of scale. Both Wal-Mart and the average consumer are able to impact the environment for the better with a mutually successful goal. Wal-Mart�s example in a research article can be a simple way to show how large politically driven corporations, even ones regarded as unethical, can still help out in the fight for a cleaner future.


\end{mla}
\end{document}
