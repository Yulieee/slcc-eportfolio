\documentclass[12pt,letterpaper]{article}
\usepackage{mla}
\usepackage{wrapfig}
\usepackage{setspace}

\begin{document}

\begin{mla}{Paul}{English}{HUMA 1100-012}{Derek
    Bitter}{\today}    
    {\textbf{Reflection Journal - Civil Disobedience}}

\section*{What does Thoreau think about government, and why?}

Thoreau begins relatively bluntly with the quoted motto, ``That government is best which governs least,'' which he modifies with a more absolute statement, ``That government is best which governs not at all.'' He goes on to explain his opinions on the matter stating that a standing government is prone to abuse and perversion, that it imposes upon the people, that it hasn't furthered any enterprise, that the people keep the country free, that the people settle the west, that the people educate. He doesn't look to highly upon the government at all. Why would he feel this way, or why does he find the government so useless?

To Thoreau, the government is a machine that seemingly gums up the works, with no obvious benefits. Anything that has been accomplished at the hands of a government could have just as easily been accomplished by the people being governed, or even worse was accomplished by the people and had taken credit for the action. To him our government is nothing more than a poorly chosen method of communicating and executing the will of the people. Whether a government is active or simply present, it can still introduce injustice into the lives of any citizen. He argues that this is the case because of the lack of conscience which a majority ruled government has, because the system that is established by a government is prone to corruption, greed, and misuse.

What alternatives would we be left with in the absence of a government? Would we be able to grow, and communicate at the same scales as we do now? Would we be able to maintain peace and understanding amongst groups of individuals. Would the kind of organization and power structures inherent in a capitalist government reform through a natural evolution of ideas and progress? I don't really know if I'd push my beliefs as far as Thoreau might when it comes to his disdain for a ruling body. I like the idea of it, but in practice I find myself somewhat satisfied with certain things that the government seems to do for us.


\end{mla}
\end{document}

