\documentclass[12pt,letterpaper]{article}
\usepackage{mla}
\usepackage{wrapfig}

\begin{document}
\begin{mla}{Paul}{English}{HUMA 1100-012}{Derek
    Bitter}{\today}{\textbf{Reflection Journal \#9 - My Brother, You Must Not Die}}

\section*{What are our responsibilities in war?}

I'm an only child, I tend to find myself in awe at the relationships that brothers and sisters have with each other, and though at times I wonder what life would be like if I were in a similar situation. When at war our family and life can significantly contribute to the direct involvement or participation in fighting. In the poem that we read by Yosano Akiko, we get a glimpse at a side of family and life that may not contribute to that participation, and I imagine that many of us might be in a similar situations, given that we might have siblings or relatives to care for, and planning or being coerced into joining a battle could be considered a flight away from responsibility. If one dies in battle, they might receive some shred of glory, but it's still a great risk, and sometimes our responsibilities require us to remain at home whether or not we believe in a particular cause. In many occasions the battle isn't at home, sometimes we have to fight to support our neighbors, only indirectly protecting our family and region. In the case where it is expected of a person, to go to war, but they are the only one to protect a lineage, or help parents and siblings, it's probably more responsible for them to stay home, though some might consider it cowardice. This is the view that is presented here when Yosano says, ``My brother, you must not die. \textbackslash You, the last born. \textbackslash Apple of our parents' eyes.''

The author pleads, as though her Brother is needed at home, as though he's already left. All that she can do is mourn the fact, and hope for a safe return. She tries to rationalize, probably correctly, that their family is not the kind of breed that should be fighting. That it is not their place, and that it's the emperor's responsibility. What will they do if he should die? Mother's might feel the same way, or wives and similar.

I think the case to be made here is that we shouldn't march blindly in support, but we should consider the cause, we should think of the reasons behind it. Maybe we're in agreement with our leaders, but if we aren't is it our responsibility to fight? Even if we are in support for the cause, should we go? It depends, but in some cases, maybe many, our primary responsibilities reside in supporting the people we love at home.

\end{mla}
\end{document}

