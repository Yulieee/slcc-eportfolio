\documentclass[12pt,letterpaper]{article}
\usepackage{mla}
\usepackage{wrapfig}
\usepackage{setspace}

\begin{document}

\begin{mla}{Paul}{English}{HUMA 1100-012}{Derek
    Bitter}{May 1, 2013}    
    {\textbf{Reflection Journal - A Chinaman's Chance}}

\section*{What does the author view as a problem with our society's views on race?}

I'm unable to fully grasp what kind of troubles might come from a minority position in the world. How it would affect my personality, my culture, or my ideas about what is right or wrong. I have many friends who are of minority descent, or of other minority groups, but I'm not. I've been very lucky in life. I've always identified myself as a white american, with little thought or hesitation into what it means for me, and how it can be a barrier for others. The author, though a minority by heritage, is not self described as a minority. He does in fact try to avoid the term asian american, and just wishes to see himself as american. Likewise he presents a case for avoiding the strong tendency we have to label and divide ourselves into subcultures. To him, he feels that the narrow tribes we group into, does nothing to help society as a whole, and certainly does nothing to improve the conditions and stereotypes of his culture, and of the other cultures we have in our country. Though it's important to understand your culture, it's not part of the american dream to be so rigidly affixed to the strange traditions, and adopted beliefs and values which are held by these small minority groups. Though he doesn't discourage this practice, he wishes that many more would take on an american identity in addition to the racial and cultural identity that they have. Who cares if you're native american, african american, mexican american, canadian american, or 3rd-generation slovak american on your dad's side, we should all embrace the beautiful culture we have around us in this country.

The author shares with us some of his experiences that has led to where he is at in life, writing this article. He share's an anecdote about his time training at an Officer Candidate School in the Marine Corps. He describes how this absurdly conservative bunch embodied one of the most tightly oriented and equal groups of men he had worked with. I suppose that through the sheer difficulty and rigor of the course that anyone who can stick around is embraced in some fashion, race and color be damned. I liked the ideas that he presented, and the refreshing encouragement to see the American Dream as though it's still an opportunity for all of us.


\end{mla}
\end{document}

