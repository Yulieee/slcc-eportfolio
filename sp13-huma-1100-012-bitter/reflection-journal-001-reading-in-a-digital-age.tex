\documentclass[12pt,letterpaper]{article}
\usepackage{mla}
\usepackage{wrapfig}

\begin{document}
\begin{mla}{Paul}{English}{HUMA 1100-012}{Derek Bitter}{\today}{\textbf{Reflection Journal \#1 - Reading In A Digital Age}}

\section{Should we work on developing sustained concentration?}

There is no doubt in my mind that sustaining concentration is one of
the most important skills any individual can possess. Without it we
could not derive or discover new mathematical concepts. We could not
solve complex processing problems in large computer systems. We would
not have some of the most interesting works of art and writing which
we have today.

As humans we have a long history of working on our state of mind.
Learning to meditate, and passively observe the wandering thoughts of
our mind. Trekking for hours on end, through unique territories.
Studying great works, as well as producing great works.

Our focus, like any other muscle or skill we possess is
something which grows with use. As children we often move so quickly
from one train of thought to another, maybe in conversations with
adults, or playing with one toy and onto another. As we grow we are
forced by the community and world around us to focus. In school, at
work, in our own hobbies. As we mature, we learn that taking the slow
route in a task can often be more worthwhile and enjoyabe. It's
presumably because of the effort we're able to put in to our
concentration.

Not everyone seems to cherish the art of focus in everything they do.
None of us are perfect of course. But it's rare to see a healthy
individual who completely neglects a thorough, careful train of
thought. For those of us who have the chance, learning to concentrate
is an opportunity that should be relished.

\end{mla}
\end{document}

