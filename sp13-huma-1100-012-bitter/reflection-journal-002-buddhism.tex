\documentclass[12pt,letterpaper]{article}
\usepackage{mla}
\usepackage{wrapfig}

\begin{document}
\begin{mla}{Paul}{English}{HUMA 1100-012}{Derek Bitter}{\today}{\textbf{Reflection Journal \#2 - Buddhism}}

\section{What does the Buddha realize?}

% the ability to cast off attachment
% enlightenment


%%%%%%
% class notes

%%%%%%
% what is enlightenment?
%  - control; emotional, spiritual
%  - ultimate goal
%    -- peace
%  - understanding meaning/purpose
%  - clear conscience
%  - no attachment

%%%%%%
% what prevents it?

%%%%%%
% why does he say ``household life is confining'' & makes it not easy
% to ``practice the holy life''?
%   - rules
%   - limits
%   - responsibilities
%   - distractions

%%%%%%
% what did the Buddha realize?
The Buddha realized that by casting off attachment amongst multiple
dimensions helped him to be more at peace with the world around him.
He realized that he could become enlightened. By continually testing
himself he learned more, and more about the world and the people
around him, about himself.

He learned that the values surrounding him may not be the values that
the others around him share. That he can avoid breathing, or avoid
food, or test himself in many ways and still the feelings and
experiences would not remain. He realizes that no matter what he does
nothing will remain, that nothing he could attach himself too would
help him in his struggles.

He learned to accept the world, and the path that goes along with it.
He derived rules and a plan to follow that would allow him to be
enlightened. He learned to share the thoughts that he had come to
accept, and to teach others that sought the same path.

% - refinement
% - aging
% - illness
% - death
% - hunger

% _Striving_
% 1. simile; wet stick, wet water, can't burn
%    - not withdrawn, body & mind
% 2. simile; wet stick, dry land, can't burn
%    - withdrawn in body only
% 3. simile; dry stick, dry land, can burn
%    - withdrawn body & mind

% _Austerities_
%    - crush mind
%    - non breathing
%    - w/o food
%    - little food
%    - trial & error

% - seclusion
% - pleasure with & without sensuality
% - directed thought

%%%%%%
% why does he think suffering is essential for enlightenment?

%    pleasure   <---   extremes    --->   pain

%%%%%%
% why did the third knowledge not remain?

% 1,2,3,4 jhana
% 
% 1st Knowledge
%    - past lives
% 2nd Knowledge
%    - other lives
% 3rd Knowledge
%    - mental fermentations
%    - stress
%    
%      +------+
%      | Mind |
%      +------+
%
%   - non-permanance
%   - no attachments

%%%%%%
% why these ``four noble truths''?

% - middle way, the eightfold path

% 1. suffering
% 2. origin of suffering
%    - craving
% 3. cessation of suffering
%    - letting go
% 4. the way leading to cessation is suffering
%    - 8-fold path

%%%%%%%%%%%%
% Why did he keep saying that, ``The pleasant feeling did not invade
% his mind or remain?''

%%%
% The elephant simile
% - attached
% - see one side
% - mistake a part for the whole
% - 

\end{mla}
\end{document}

